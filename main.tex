\documentclass{beamer}
\usepackage{graphicx} % Required for inserting images

\usepackage[utf8]{inputenc}
\usepackage[T1]{fontenc}
\usepackage{lmodern}
\usepackage{amsmath,amssymb}
\usepackage{microtype}
\usepackage{ellipsis}
%\usepackage[ngerman]{babel}
\let\openbox\undefined
\usepackage{mathtools}
\usepackage{enumitem}
\let\openbox\undefined
\usepackage{amsthm}
\usepackage{thmtools}
\usepackage{graphicx}
\usepackage{stmaryrd}
\usepackage{tikz}
\usetikzlibrary{positioning}
\usepackage{algpseudocode}
\usepackage[absolute,overlay]{textpos}
\usepackage{url}
\usepackage[
backend=biber,
style=numeric,
]{biblatex}
\addbibresource{references.bib}
\usepackage[normalem]{ulem}

\declaretheoremstyle[
    headpunct=,
    spacebelow=2em,
    spaceabove=1em,
    postheadspace=\newline,
    ]{aufgabe}
\declaretheorem[style=aufgabe]{aufgabe}
\numberwithin{equation}{aufgabe}
\addtolength\jot{1ex}

\newtheorem{proposition}{Proposition}
\renewcommand\qedsymbol{$\square$}

\newcommand\R{\mathbb R}
\newcommand\Z{\mathbb Z}
\newcommand\N{\mathbb N}
\newcommand\C{\mathbb C}
\newcommand{\Q}{\mathbb Q}
\newcommand{\F}{\mathbb{F}}
\newcommand{\ass}{\underline{Assume:}  }
\newcommand{\zz}{\underline{t.s.:}  }

\usetheme[compress]{Berlin}
\setbeamertemplate{footline}[frame number]{}
\setbeamertemplate{navigation symbols}{}
\setbeamertemplate{footline}{}

\makeatletter
\beamer@theme@subsectionfalse%
\makeatother


\title{The GAPic Package}
\subtitle{}
\author{Lukas Schnelle}
\date{GAPDays Spring 2024}

\begin{document}

% remove dots on first slide
\frame[plain]{\titlepage}

\section{Definitions}
\begin{frame}
    \begin{definition}\label{def:triangular-comp}
    	Let $X_0, X_1, X_2 \subset \N_0$ be finite subsets and \pause $\prec$ a symmetric relation between the sets called \emph{incidence}. \pause \\
    	We call $(\prec, X_0, X_1, X_2)$ \emph{triangular complex} if
    	\begin{enumerate}[label=(\roman*)] \pause 
    		\item The relation $\prec$ is transitive, i.e. $\forall v \in X_0, e \in X_1, f \in X_2$:
    		$$(v \prec e) \wedge (e \prec f) \Rightarrow v \prec f.$$
    		\item \pause $\forall e \in X_1 \exists !\,  v_1 \neq v_2 \in X_0$ such that $v_1 \prec e, v_2 \prec e$.\pause 
    		\item $\forall e \in X_1 \exists \, f \in X_2$ such that $e \prec f$. \pause 
    		\item $\forall f \in X_2 \exists ! \, v_1 \neq v_2 \neq v_3 \in X_0$ such that $v_1, v_2, v_3 \prec f$. \pause 
            \item $\forall f \in X_2 \exists ! \, e_1 \neq e_2 \neq e_3 \in X_1$ such that $e_1, e_2, e_3 \prec f$. \pause 
    	\end{enumerate}
    	The elements in $X_0$ are called \emph{vertices}, the elements in $X_1$ are called \emph{edges} and the elements in $X_2$ are called \emph{faces}.
    \end{definition}
\end{frame}

\begin{frame}
    \begin{definition}
        Let $(\prec, X_0, X_1, X_2)$ be a triangular complex.\\
        Then we call $(\prec, X_0, X_1, X_2)$ \emph{simplicial surface} if 
        \begin{enumerate}[label=(\roman*)]
            \item $\forall e \in X_1 : | \{ f \in X_2 \mid e \prec f \} | \leq 2 $
            \item $\forall v \in X_0 : | \{ f \in X_2 \mid v \prec f \} | < \infty$
            \item $\forall v \in X_0: $ there is an ordering of the $e_i, f_j \prec v$ such that 
            $$
                e_1 \prec f_1 \prec e_2 \prec f_2 \prec \dots \prec f_{n-1} \prec e_{n} \prec f_{n} \prec e_1
            $$
        \end{enumerate}
        the last condition is called the \emph{umbrella condition}.
    \end{definition}
\end{frame}

\begin{frame}
    \begin{example}
        \includegraphics[width=0.48\textwidth]{images/tet.png}
        \uncover<3->{
        \includegraphics[width=0.48\textwidth]{images/complex3.png}}\\
        \uncover<2->{Simplicial Surface} \hspace{75pt} \uncover<4>{Triangular complex}
    \end{example}
\end{frame}

\begin{frame}
    \begin{definition}\label{def:embedding}
        Let $(\prec, X_0, X_1, X_2)$ be a triangular complex. \\ \pause
        Then we define an \emph{embedding} of $(\prec, X_0, X_1, X_2)$ as a map 
        $$c: X_0 \to \R^3.$$ 
        The image of $v \in X_0$ is called \emph{coordinate of $v$}.
    \end{definition}
    \pause
    \begin{example}
        \centering
        \includegraphics[width=0.3\textwidth]{images/tet.png}\\
        Is an embedded triangular complex.
    \end{example}
\end{frame}

\section{SimplicialSurfaces Package}
\begin{frame}{Simplicial Surfaces Package}
    \begin{itemize}[label=-]
        \item Has functionality for displaying surfaces
        \begin{itemize}[label=-]
            \item Generates a .html file
            \item Uses three.js
        \end{itemize}
    \end{itemize}
    \pause
    \begin{example}[Number 2.1 and 2.2 from \cite{ico}]
        \includegraphics[width=0.4\textwidth]{images/ico-old.png}
        \includegraphics[width=0.4\textwidth]{images/ico2-1_old.png}
    \end{example}
\end{frame}

\begin{frame}
    \begin{exampleblock}{Fachpraktikum}
        Was a project with the goal to improve the visualizations by adding shading/local lighting
        \pause
        After some work it turns out: central class used in implementation is deprecated.
        \includegraphics[width=1\textwidth]{images/three-deprecated.png}
    \end{exampleblock}
\end{frame}

\begin{frame}{Goal}
    \begin{exampleblock}{Goal}
        Implement shading in the visualizations of the Simplicial Surface package.
    \end{exampleblock}
    \pause
    First Approach: Implement directly\\
    $\xrightarrow{}$ Learn how the output is generated\\
    Uses a class called \texttt{THREE.Geometry} \\
    
\end{frame}

\begin{frame}{Workflow}
    \textbf{But:} In newer revisions of three.js shading is already implemented.\\
    $\xrightarrow{}$ After some promisiong tests: Decided to rewrite the entire function.
\end{frame}

\begin{frame}{Demo}
    We need to switch to the browser for this.\\
    For one example we use \cite{ico}
    
\end{frame}

\begin{frame}{Results}
    \begin{block}{ToDos}
        \begin{itemize}[label=-]
            \item Some of the functions are not yet implemented again
        \end{itemize}
    \end{block}
\pause
    \begin{exampleblock}{Advantages}
        \begin{itemize}[label=-]
            \item New security requirements of javascript and modern browsers: need to load the code from server $\xrightarrow{}$ way smaller file sizes\\
                (for small examples 9kB vs. 539kB)
                \pause
            \item More efficient Animations, faster loading, fewer memory\\
                (Demo in Browser)
                \pause
            \item Also works for triangular complexes\\
            $\xrightarrow{}$Does not depend on incidence structure for visualization\\
            (Demo in Browser)
        \end{itemize}
    \end{exampleblock}
\end{frame}
\begin{frame}
    \begin{block}{Future}
        \begin{itemize}[label=-]
            \item More functions in the GUI, e.g. \\
            \begin{itemize}[label=-]
                \item Turning the vertices on and off
                \item Changing location of a vertex on the fly
            \end{itemize}
            \item More options materials\\
                e.g. Color dependent on the normal of the polygon\\
                (Demo in browser)
            \item Intersection planes \\
                (Demo in browser)    
        \end{itemize}
    \end{block}
\end{frame}

\begin{frame}
    \textbf{\Large Thank You for your attention}\\
    \bigskip
    Are there Questions?
    \bigskip
    
    \printbibliography
\end{frame}

\end{document}
